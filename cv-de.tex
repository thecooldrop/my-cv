%-------------------------
% Lebenslauf in Latex
% Autor: Jake Gutierrez
% Basierend auf: https://github.com/sb2nov/resume
% Lizenz: MIT
%------------------------
\documentclass[a4paper,12pt]{article}

\usepackage{latexsym}
\usepackage[empty]{fullpage}
\usepackage{titlesec}
\usepackage{marvosym}
\usepackage[usenames,dvipsnames]{color}
\usepackage{verbatim}
\usepackage{enumitem}
\usepackage[hidelinks]{hyperref}
\usepackage{fancyhdr}
\usepackage[german]{babel}
\usepackage{tabularx}
\usepackage{svg}
\input{glyphtounicode}

\pagestyle{fancy}
\fancyhf{} % alle Kopf- und Fußzeilenfelder löschen
\fancyfoot{}
\renewcommand{\headrulewidth}{0pt}
\renewcommand{\footrulewidth}{0pt}

% Ränder anpassen
\addtolength{\oddsidemargin}{-0.5in}
\addtolength{\evensidemargin}{-0.5in}
\addtolength{\textwidth}{1in}
\addtolength{\topmargin}{-.5in}
\addtolength{\textheight}{1.0in}

\urlstyle{same}

\raggedbottom
\raggedright
\setlength{\tabcolsep}{0in}

% Formatierung der Abschnitte
\titleformat{\section}{
	\vspace{-4pt}\scshape\raggedright\large
}{}{0em}{}[\color{black}\titlerule \vspace{-5pt}]

% Sicherstellen, dass die generierte PDF maschinenlesbar/ATS-analysierbar ist
\pdfgentounicode=1

%-------------------------
% Benutzerdefinierte Befehle
\newcommand{\resumeItem}[1]{
	\item\small{
		{#1 \vspace{-2pt}}
	}
}

\newcommand{\resumeSubheading}[4]{
	\vspace{-2pt}\item
	\begin{tabular*}{0.97\textwidth}[t]{l@{\extracolsep{\fill}}r}
		\textbf{#1} & #2 \\
		\textit{\small#3} & \textit{\small #4} \\
	\end{tabular*}\vspace{-7pt}
}

\newcommand{\resumeSubheadingSingleLine}[2]{
	\vspace{-2pt}\item
	\begin{tabular*}{0.97\textwidth}[t]{l@{\extracolsep{\fill}}r}
		\textbf{#1} & #2
	\end{tabular*}\vspace{-7pt}
}

\newcommand{\resumeSubSubheading}[2]{
	\item
	\begin{tabular*}{0.97\textwidth}{l@{\extracolsep{\fill}}r}
		\textit{\small#1} & \textit{\small #2} \\
	\end{tabular*}\vspace{-7pt}
}

\newcommand{\resumeProjectHeading}[2]{
	\item
	\begin{tabular*}{0.97\textwidth}{l@{\extracolsep{\fill}}r}
		\small#1 & #2 \\
	\end{tabular*}\vspace{-7pt}
}

\newcommand{\resumeSubItem}[1]{\resumeItem{#1}\vspace{-4pt}}

\renewcommand\labelitemii{$\vcenter{\hbox{\tiny$\bullet$}}$}

\newcommand{\resumeSubHeadingListStart}{\begin{itemize}[leftmargin=0.15in, label={}]}
	\newcommand{\resumeSubHeadingListEnd}{\end{itemize}}
\newcommand{\resumeItemListStart}{\begin{itemize}}
	\newcommand{\resumeItemListEnd}{\end{itemize}\vspace{-5pt}}

\usepackage{fontawesome5}

\begin{document}
	
	\begin{center}
		\textbf{\Huge \scshape Vanio Begic} \\ \vspace{2pt} 
		Vogesenstr. 25, 76287 Rheinstetten $\cdot$ \small +49 176 244 15137 \\ \vspace{2pt}
		\faEnvelope \ \href{mailto:vanio.begic123@gmail.com.com}{\underline{vanio.begic123@gmail.com}} $|$ 
		\faLinkedin \ \href{https://linkedin.com/in/vaniobegic}{\underline{linkedin.com/in/vaniobegic}} $|$
		\faGithub \ \href{https://github.com/thecooldrop}{\underline{github.com/thecooldrop}}
	\end{center}
	
	\rule[5pt]{\textwidth}{2pt}
	
	Software-Ingenieur spezialisiert auf Entwicklung und Management von Java-Backend-Anwendungen für den Betrieb 
	in containerisierten Cloud-Umgebungen. Erfahren in DevOps-Methodik mit nachgewiesenen Erfolgen 
	bei der Automatisierung von Integrations- und Bereitstellungsaufgaben. Open-Source-Enthusiast und Fan aller 
	Cloud-nativen Technologien.
	
	%-----------BILDUNG-----------
	\section{Bildung}
	\resumeSubHeadingListStart
	\resumeSubheading
	{Karlsruher Institut für Technologie}{Karlsruhe, Deutschland}
	{Bachelor in Elektrotechnik und Informationstechnik}{Okt. 2016 -- Nov. 2019}
	\resumeSubheading
	{Karlsruher Institut für Technologie}{Karlsruhe, Deutschland}
	{Feststellungsprüfung}{Aug. 2015 -- Aug. 2016}
	\resumeSubHeadingListEnd
	
	
	%-----------BERUFSERFAHRUNG-----------
	\section{Berufserfahrung}
	\resumeSubHeadingListStart
		\resumeSubheading
		{Senior Software Ingenieur (Stufe C)}{Okt. 2024 -- Heute}{Capgemini}{Karlsruhe, Deutschland}
		\resumeItemListStart
			\resumeItem{Mitglied eines funktionsübergreifenden Teams mit Fokus auf DevOps-Themen im Kontext neuer und anspruchsvoller Kundenprojekte.}
			\resumeItem{Evangelist für die Etablierung von Innersource innerhalb von Capgemini.}
			\resumeItem{Entwicklung eines hausinternen agentischen Frameworks für die automatisierte Transformation von Anwendungen von Legacy- zu modernen Technologien.}
			\resumeItem{Anwendung des agentischen Frameworks im Kundenkontext für die Migration einer Anwendungsoberfläche von Struts 1 zu React.}
			\resumeItem{Entwicklung einer E-Commerce-Plattform für den B2B-Verkauf von Immobiliendaten im Bankensektor.}
			\resumeItem{\textbf{Eingesetzte Fähigkeiten: }{Java, Cypress, Cucumber, JUnit 5, Spring-Ökosystem, Gradle, Portainer, Docker/Container, Python, CrewAI, Prompt Engineering}}
		\resumeItemListEnd
		\resumeSubSubheading{People Development Lead}{März 2025 -- Heute}
			\resumeItemListStart
				\resumeItem{Formale Führung und Management eines Teams von Software-Ingenieuren als Teil der Capgemini-Matrixorganisation.}
			\resumeItemListEnd

		\resumeSubheading
		{Lead Software Ingenieur (Stufe B)}{Apr. 2022 -- Okt. 2024}
		{Capgemini}{Karlsruhe, Deutschland}
			\resumeItemListStart
				\resumeItem{Mitglied eines funktionsübergreifenden Teams mit dem Auftrag, neue Engagements zu etablieren, zu stabilisieren und zu formen, mit Fokus auf die Gestaltung der Architektur und Teaminteraktionen, um bekannte Anti-Patterns bei der projektorientierten Softwarebereitstellung für große Unternehmenskunden zu vermeiden.}
				\resumeItem{Entwicklung einer Microservice-Landschaft, die über RabbitMQ kommunizieren und mit zustandsbehafteter Verarbeitung von Daten agieren, um Verbraucherverträge mit Vereinbarungen von Anbietern digitaler Dienste abzugleichen.}
				\resumeItem{Integrations-, Contract- und Behaviour-Driven-Tests einer Microservice-Landschaft mit Pact, Cucumber und JUnit5.}
				\resumeItem{Entwicklung einer Frontend-Anwendung in Angular für die Beantragung von Zugriff auf komplexe Datenverarbeitung unter den Einschränkungen der DSGVO im öffentlichen Gesundheitssektor.}
				\resumeItem{Entwicklung eines Backends mit integrierter komplexer Zustandsmaschine zur Verwaltung des Lebenszyklus von Anfragen für den Zugriff auf Daten im öffentlichen Gesundheitssektor.}
				\resumeItem{\textbf{Eingesetzte Fähigkeiten: }{Java, Typescript, Pact, JUnit5, Cypress, Cucumber, Jest, Hibernate, Spring-Ökosystem, Docker/Container, Kubernetes, Helm, Kustomize, ArgoCD, Rancher, Gitlab CI, Bash, RabbitMQ, REST APIs}}
				\resumeItem{\textbf{Eingesetzte Methoden: }{Trunk-Based Development, Test-Driven Development, Kanban, Scrum, Value-Stream Mapping, WIP-Begrenzung, Pair Programming}}
			\resumeItemListEnd
	
		\resumeSubheading{Software Ingenieur (Stufe A)}{Jan. 2020 -- Apr. 2022 }{Capgemini}{Karlsruhe, Deutschland}
			\resumeItemListStart
				\resumeItem{Migration einer Legacy-Java-Enterprise-Backend-Anwendung von einem monolithisch integrierten Dienst auf einem Mainframe zu einer cloud-fähigen 12-Faktor-Anwendung, die in AWS EKS läuft.}
				\resumeItem{Entkopplung eines Projekts von der Legacy-Wasserfall-Entwicklungsmethodik und Übergang zur DevOps-Methodik.}
				\resumeItem{Entwicklung und Verwaltung der Cloud-Infrastruktur über Terraform, Jenkins-Pipelines und Konfigurationen-als-Code, sowie deklarative Beschreibung von Deployments in verschiedenen Umgebungen als Teil der Bemühungen, zu GitOps-artigen Deployments und Betriebsmanagement zu migrieren.}
				\resumeItem{Leitung der Einführung von Cloud-Technologien, um die Leistungsfähigkeit von Kubernetes zu nutzen und moderne Tools in die Projekte zu integrieren.}
				\resumeItem{\textbf{Eingesetzte Fähigkeiten: }{Java, Spring Batch, Java EE, Maven, Mockito, JUnit5, Jenkins, Kubernetes, Terraform, AWS, Docker, Helm, IntelliJ}}
			\resumeItemListEnd
	
	\resumeSubheading
	{Praktikant - Autonome Robotersysteme}{Okt. 2018 -- Nov. 2019}
	{SEW Eurodrive}{Bruchsal, Deutschland}
	\resumeItemListStart
	\resumeItem{Implementierung neuartiger adaptiver Algorithmen zur Verfolgung ausgedehnter Ziele mit variabler Ausdehnung basierend auf Radarmessungen.}
	\resumeItem{Mathematische Herleitung einer Variation des GM-PHD-Algorithmus, die variable Zielausdehnung berücksichtigt und eine gleichmäßige Verteilung neu beobachteter Ziele über den Zustandsraum annimmt.}
	\resumeItem{Mitarbeit bei der Bewertung von Synergien zwischen SLAM (Simultaneous Localization and Mapping) und Algorithmen zur Verfolgung mehrerer Ziele für automatisierte Roboterplattformen in Innenräumen.}
	\resumeItemListEnd
	
	\resumeSubHeadingListEnd
	
	
	%-----------PROJEKTE-----------
	% \section{Projekte}
	% \resumeSubHeadingListStart
	% \resumeProjectHeading
	% {\textbf{Gitlytics} $|$ \emph{Python, Flask, React, PostgreSQL, Docker}}{Juni 2020 -- Heute}
	% \resumeItemListStart
	% \resumeItem{Entwicklung einer Full-Stack-Webanwendung mit Flask für eine REST-API und React als Frontend}
	% \resumeItem{Implementierung von GitHub OAuth, um Daten aus den Repositories der Benutzer zu erhalten}
	% \resumeItem{Visualisierung von GitHub-Daten zur Darstellung der Zusammenarbeit}
	% \resumeItem{Nutzung von Celery und Redis für asynchrone Aufgaben}
	% \resumeItemListEnd
	% \resumeProjectHeading
	% {\textbf{Simple Paintball} $|$ \emph{Spigot API, Java, Maven, Git}}{Mai 2018 -- Mai 2020}
	% \resumeItemListStart
	% \resumeItem{Entwicklung eines Minecraft-Server-Plugins zur Unterhaltung von Kindern in der Freizeit für einen früheren Job}
	% \resumeItem{Veröffentlichung des Plugins auf Websites mit über 2.000 Downloads und einer durchschnittlichen Bewertung von 4,5/5 Sternen}
	% \resumeItem{Implementierung von Continuous Delivery mit TravisCI zum Erstellen des Plugins bei neuen Releases}
	% \resumeItem{Zusammenarbeit mit Minecraft-Server-Administratoren zur Vorschlagung von Funktionen und zum Einholen von Feedback zum Plugin}
	% \resumeItemListEnd
	% \resumeSubHeadingListEnd
	

	\section{Zertifizierungen}
	\begin{itemize}[leftmargin=0.15in, label={}]
		\small {
			\item Dez. 2024 \quad \textbf{Certified Kubernetes Administrator}, Linux Foundation
		}
	\end{itemize}
	
	%
	%-----------FÄHIGKEITEN-----------
	\section{Fähigkeiten}
	\begin{itemize}[leftmargin=0.15in, label={}]
		\small{
			\item{
				\textbf{Sprachen}{: Deutsch (C1), Englisch (fließend), Bosnisch/Kroatisch/Serbisch (Muttersprache)} \\
				\textbf{Software-Engineering}{: Softwarearchitektur auf IT-Systemebene in Cloud-Umgebungen (Schwerpunkt AWS),
					Anwendungsarchitektur (Schwerpunkt SOLID und hexagonal), Test-Driven Development, Migration von
					Legacy-Anwendungen in die Cloud, Consumer-Driven Contract Testing, Behavior Driven Development} \\
					\textbf{Frameworks}{: Spring, Mockito, JUnit, Hibernate, Liquibase, Flyway, Cucumber, Pact, Angular, React, Jest, Cypress} \\
					\textbf{Entwicklertools}{: Git, Docker, AWS, VS Code, IntelliJ, Eclipse} \\
					\textbf{DevOps}{: Grafana, Prometheus, Kubernetes, Helm, Kustomize, Jib, Kaniko, Jenkins, Gitlab CI, Shell Scripting, Terraform, Serverless Framework, DaggerCI, Github Actions}
			}}
		\end{itemize}
	
		\section{Open-Source-Beiträge}
			\resumeSubHeadingListStart
				\resumeSubheadingSingleLine{\includesvg{hyperlink} \href{https://github.com/spring-projects/spring-boot/pulls/thecooldrop}{Spring Boot}}{}{}{}
					\resumeItemListStart[label=\includesvg{merged}]
						\resumeItem{\href{https://github.com/spring-projects/spring-boot/pull/44187}{Dokumentation aller verfügbaren Testcontainers-Integrationen \#44187}}
						\resumeItem{\href{https://github.com/spring-projects/spring-boot/pull/44500}{Entfernung der Unterstützung für URLConnectionSender \#44500}}
						\resumeItem{\href{https://github.com/spring-projects/spring-boot/pull/44499}{Hinzufügung der Unterstützung für OpenTelemetry's service.namespace \#44499}}
					\resumeItemListEnd
				\resumeSubheadingSingleLine{\includesvg{hyperlink} \href{https://github.com/argoproj/argo-cd/pulls/thecooldrop}{ArgoCD}}{}{}{}
					\resumeItemListStart[label=\includesvg{merged}]
						\resumeItem{\href{https://github.com/argoproj/argo-cd/pull/21093}{chore: Verbesserung der Dokumentation bezüglich der Auswahl von Anwendungen durch Sync Windows (\#20971) \#21093}}
						\resumeItem{\href{https://github.com/argoproj/argo-cd/pull/19000}{chore: Klärung der Bedeutung von Secret-Exfiltration über ApplicationSet (Issue \#18560) \#19000}}
						\resumeItem{\href{https://github.com/argoproj/argo-cd/pull/18990}{fix: Baumausgabe mit App-Selektor-Zugriffsverweigerung \#18990}}
						\resumeItem{\href{https://github.com/argoproj/argo-cd/pull/18860}{fix: Behebung eines Problems beim Starten von Telepresence für Remote-Debugging (Schließt \#18859) \#18860}}
					\resumeItemListEnd
				\resumeSubheadingSingleLine{\includesvg{hyperlink} \href{https://github.com/keycloak/keycloak/pulls/thecooldrop}{Keycloak}}{}{}{}
					\resumeItemListStart[label=\includesvg{merged}]
					\resumeItem{\href{https://github.com/keycloak/keycloak/pull/25021}{Ersetzung von statischem Text durch Referenz zur Übersetzung in der Anwendungsseite \#24527 \#25021}}
					\resumeItemListEnd
			\resumeSubHeadingListEnd
		
		%-------------------------------------------
	\end{document}
	