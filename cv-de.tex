\documentclass[a4paper,12pt]{article}

\usepackage{latexsym} \usepackage[empty]{fullpage} \usepackage{titlesec} \usepackage{marvosym} \usepackage[usenames,dvipsnames]{color} \usepackage{verbatim} \usepackage{enumitem} \usepackage[hidelinks]{hyperref} \usepackage{fancyhdr} \usepackage[german]{babel} \usepackage{tabularx} \input{glyphtounicode}

\pagestyle{fancy}
\fancyhf{} 
\fancyfoot{}
\renewcommand{\headrulewidth}{0pt}
\renewcommand{\footrulewidth}{0pt}

% Adjust margins
\addtolength{\oddsidemargin}{-0.5in}
\addtolength{\evensidemargin}{-0.5in}
\addtolength{\textwidth}{1in}
\addtolength{\topmargin}{-.5in}
\addtolength{\textheight}{1.0in}

\urlstyle{same}

\raggedbottom
\raggedright
\setlength{\tabcolsep}{0in}

\titleformat{\section}{
	\vspace{-4pt}\scshape\raggedright\large
}{}{0em}{}[\color{black}\titlerule \vspace{-5pt}]

\pdfgentounicode=1

%-------------------------
% Custom commands
\newcommand{\resumeItem}[1]{
	\item\small{
		{#1 \vspace{-2pt}}
	}
}

\newcommand{\resumeSubheading}[4]{
	\vspace{-2pt}\item
	\begin{tabular*}{0.97\textwidth}[t]{l@{\extracolsep{\fill}}r}
		\textbf{#1} & #2 \\
		\textit{\small#3} & \textit{\small #4} \\
	\end{tabular*}\vspace{-7pt}
}

\newcommand{\resumeSubSubheading}[2]{
	\item
	\begin{tabular*}{0.97\textwidth}{l@{\extracolsep{\fill}}r}
		\textit{\small#1} & \textit{\small #2} \\
	\end{tabular*}\vspace{-7pt}
}

\newcommand{\resumeProjectHeading}[2]{
	\item
	\begin{tabular*}{0.97\textwidth}{l@{\extracolsep{\fill}}r}
		\small#1 & #2 \\
	\end{tabular*}\vspace{-7pt}
}

\newcommand{\resumeSubItem}[1]{\resumeItem{#1}\vspace{-4pt}}

\renewcommand\labelitemii{$\vcenter{\hbox{\tiny$\bullet$}}$}

\newcommand{\resumeSubHeadingListStart}{\begin{itemize}[leftmargin=0.15in, label={}]}
	\newcommand{\resumeSubHeadingListEnd}{\end{itemize}}
\newcommand{\resumeItemListStart}{\begin{itemize}}
	\newcommand{\resumeItemListEnd}{\end{itemize}\vspace{-5pt}}

\usepackage{fontawesome5}

\begin{document}

\begin{center}
	\textbf{\Huge \scshape Vanio Begic} \\ \vspace{2pt} 
	Vogesenstr. 25, 76287 Rheinstetten $\cdot$ \small +49 176 244 15137 \\ \vspace{2pt}
	\faEnvelope \ \href{mailto:vanio.begic123@gmail.com.com}{\underline{vanio.begic123@gmail.com}} $|$ 
	\faLinkedin \ \href{https://linkedin.com/in/vaniobegic}{\underline{linkedin.com/in/vaniobegic}} $|$
	\faGithub \ \href{https://github.com/thecooldrop}{\underline{github.com/thecooldrop}}
\end{center}

\rule[5pt]{\textwidth}{2pt}
Senior Software Ingenieur spezialisiert in Entwicklung und Wartung von Java Backend Anwendungen für Betrieb in containerisierten Cloud-Umgebungen.
Erfahren mit DevOps Methodologie mit demonstrierbarem Erfolg in Automatisierung von Integrations- und Auslieferungsprozessen.
Open-Source-Enthusiast und Fan von allem Cloud-Native.

%-----------EDUCATION-----------
\section{Bildungsweg}
\resumeSubHeadingListStart
\resumeSubheading
{Karlsruher Institut für Technologie}{Karlsruhe, Deutschland}
{Bachelor of Elektrotechnik und Informationstechnik}{Oktober 2016 -- November 2019}
\resumeSubheading
{Studienkolleg am Karlsruher Institut für Technologie}{Karlsruhe, Deutschland}
{Feststellungsprüfung}{August 2015 -- August 2016}
\resumeSubheading{Gymnasium Ismet Mujezinovic}{Tuzla, Bosnien und Herzegowina}{Abitur}{September 2011 -- Juni 2015}
\resumeSubheading{Grundschule Pazar}{Tuzla, Bosnien und Herzegowina}{Grundschulabschluss}{September 2007 -- Juni 2011}
\resumeSubheading{Grundschule Lukavac}{Lukavac, Bosnien und Herzegowina}{Fortsetzung in anderer Stadt}{September 2003 -- Juni 2007}
\resumeSubHeadingListEnd

%-----------EXPERIENCE-----------
\section{Berufserfahrung}
\resumeSubHeadingListStart
\resumeSubheading
{Senior Software Ingenieur}{Oktober 2024 -- laufend}{Capgemini}{Karlsruhe, Deutschland}
\resumeItemListStart
	\resumeItem{Mitglied eines funktionsübergreifenden Teams mit Fokus auf DevOps-Themen im Kontext neuer und anspruchsvoller Kundenprojekte.}
	\resumeItem{Community Evangelist für die Etablierung von Innersource innerhalb von Capgemini.}
\resumeItemListEnd
\resumeSubheading
{Lead Software Ingenieur}{April 2022 -- Oktober 2024}
{Capgemini}{Karlsruhe, Deutschland}
\resumeItemListStart
\resumeItem{Mitglied eines funktionsübergreifenden Teams mit dem Auftrag, neue Engagements zu etablieren, zu stabilisieren und zu formen, mit Schwerpunkt auf der Gestaltung der Architektur und der Teaminteraktionen, um bekannte Lieferantimuster in der projektorientierten Softwarebereitstellung für große Unternehmenskunden zu vermeiden.}
\resumeItemListEnd

\resumeSubheading{Software Ingenieur}{Januar 2020 -- April 2022 }{Capgemini}{Karlsruhe, Deutschland}
\resumeItemListStart
	\resumeItem{Migration von veralteten Enterprise-Java Backend Anwendungen von einer monolithischen Architektur zu einer cloud-fähigen 12-Faktor-Anwendung, die in AWS EKS läuft.}
\resumeItem{Entkopplung eines Projekts von der Legacy-Wasserfall-Entwicklungsmethodik und Migration zu einer cloud-fähigen DevOps-Methodik.}
\resumeItem{Entwicklung und Verwaltung von Infrastruktur über Terraform, Jenkins-Pipelines und Konfigurationen als Code, sowie deklarative Beschreibung von Deployments in verschiedenen Umgebungen als Teil der Bemühungen, zu GitOps-artigen Deployments und Betriebsmanagement zu migrieren.}
\resumeItem{Leitung der Einführung von Cloud-Technologien, um die Leistungsfähigkeit von Kubernetes zu nutzen und modernste Tools in die Projekte zu integrieren.}
\resumeItemListEnd

\resumeSubheading
{Praktikant - Autonome Robotersysteme}{Okt. 2018 -- Nov. 2019}
{SEW Eurodrive}{Bruchsal, Deutschland}
\resumeItemListStart
\resumeItem{Implementierung neuartiger adaptiver Algorithmen zur Verfolgung ausgedehnter Ziele mit variabler Ausdehnung basierend auf Radarmessungen.}
\resumeItem{Mathematische Herleitung einer Variation des GM-PHD-Algorithmus, die variable Zielausdehnung berücksichtigt und eine gleichmäßige Verteilung neu beobachteter Ziele über den Zustandsraum annimmt.}
\resumeItem{Mitarbeit bei der Bewertung von Synergien zwischen SLAM (Simultaneous Localization and Mapping) und Algorithmen zur Verfolgung mehrerer Ziele für automatisierte Roboterplattformen in Innenräumen.}
\resumeItemListEnd

\resumeSubHeadingListEnd

\section{Zertifizierungen}
\begin{itemize}[leftmargin=0.15in, label={}]
	\small {
		\item Dez. 2024 \quad \textbf{Certified Kubernetes Administrator}, Linux Foundation
	}
\end{itemize}

%-----------TECHNICAL SKILLS-----------
\section{Technische Fähigkeiten}
\begin{itemize}[leftmargin=0.15in, label={}]
	\small{
		\item{
			\textbf{Software Engineering}{: Softwarearchitektur auf IT-Systemebene in Cloud-Umgebungen (Fokus auf AWS),
				Anwendungsarchitektur (Fokus auf SOLID und hexagonal), Test-Driven Development, Migration von
				Legacy-Anwendungen in die Cloud, Consumer-Driven Contract Testing, Behavior Driven Development} \\
			\textbf{Frameworks}{: Spring, Mockito, JUnit, Hibernate, Liquibase, Flyway, Cucumber, Pact, Angular, React, Jest, Cypress} \\
			\textbf{Entwicklertools}{: Git, Docker, AWS, VS Code, IntelliJ, Eclipse} \\
			\textbf{DevOps}{: Grafana, Prometheus, Kubernetes, Helm, Kustomize, Jib, Kaniko, Jenkins, Gitlab CI, Shell Scripting, Terraform, Serverless Framework, DaggerCI, Github Actions}
	}}
\end{itemize}

\end{document}
%-------------------------------------------